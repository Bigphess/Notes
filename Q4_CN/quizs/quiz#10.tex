\documentclass{article}

\usepackage{fancyhdr}
\usepackage{extramarks}
\usepackage{amsmath}
\usepackage{amsthm}
\usepackage{amsfonts}
\usepackage{tikz}
\usepackage{graphicx} %插入图片的宏包
\usepackage{float} %设置图片浮动位置的宏包
\usepackage{pythonhighlight}
% \usepackage{subfigure} %插入多图时用子图显示的宏包
% \usepackage[plain]{algorithm}
% \usepackage{algpseudocode}

% \usetikzlibrary{automata,positioning}

%
% Basic Document Settings
%

\topmargin=-0.45in
\evensidemargin=0in
\oddsidemargin=0in
\textwidth=6.5in
\textheight=9.0in
\headsep=0.25in

\linespread{1.1}

\pagestyle{fancy}
\lhead{\hmwkAuthorName}
\chead{\hmwkClass\ : \hmwkTitle}
\rhead{\firstxmark}
\lfoot{\lastxmark}
\cfoot{\thepage}

\renewcommand\headrulewidth{0.4pt}
\renewcommand\footrulewidth{0.4pt}

\setlength\parindent{0pt}


%代码格式设置



%
% Create Problem Sections
%

\newcommand{\enterProblemHeader}[1]{
    \nobreak\extramarks{}{Problem \arabic{#1} continued on next page\ldots}\nobreak{}
    \nobreak\extramarks{Problem \arabic{#1} (continued)}{Problem \arabic{#1} continued on next page\ldots}\nobreak{}
}

\newcommand{\exitProblemHeader}[1]{
    \nobreak\extramarks{Problem \arabic{#1} (continued)}{Problem \arabic{#1} continued on next page\ldots}\nobreak{}
    \stepcounter{#1}
    \nobreak\extramarks{Problem \arabic{#1}}{}\nobreak{}
}

\setcounter{secnumdepth}{0}
\newcounter{partCounter}
\newcounter{homeworkProblemCounter}
\setcounter{homeworkProblemCounter}{1}
\nobreak\extramarks{Problem \arabic{homeworkProblemCounter}}{}\nobreak{}

%
% Homework Problem Environment
%
% This environment takes an optional argument. When given, it will adjust the
% problem counter. This is useful for when the problems given for your
% assignment aren't sequential. See the last 3 problems of this template for an
% example.
%
\newenvironment{homeworkProblem}[1][-1]{
    \ifnum#1>0
        \setcounter{homeworkProblemCounter}{#1}
    \fi
    \section{Problem \arabic{homeworkProblemCounter}}
    \setcounter{partCounter}{1}
    \enterProblemHeader{homeworkProblemCounter}
}{
    \exitProblemHeader{homeworkProblemCounter}
}

%
% Homework Details
%   - Title
%   - Due date
%   - Class
%   - Section/Time
%   - Instructor
%   - Author
%

\newcommand{\hmwkTitle}{Quiz\ \#10}
\newcommand{\hmwkDueDate}{Jan 20th, 2019}
\newcommand{\hmwkClass}{Complex Networks}
\newcommand{\hmwkClassTime}{Section A}
% \newcommand{\hmwkClassInstructor}{Professor Isaac Newton}
\newcommand{\hmwkAuthorName}{\textbf{RUOPENG XU} }
\newcommand{\hmwkAuthorNum}{\textbf{18M38179} }

%
% Title Page
%

\title{
    \vspace{2in}
    \textmd{\textbf{\hmwkClass:\ \hmwkTitle}}\\
    \normalsize\vspace{0.1in}\small{Due\ on\ \hmwkDueDate\ }\\
    % \vspace{0.1in}\large{\textit{\hmwkClassInstructor\ \hmwkClassTime}}
    \vspace{3in}
}

\author{\hmwkAuthorName\\ \hmwkAuthorNum}
\date{}

\renewcommand{\part}[1]{\textbf{\large Part \Alph{partCounter}}\stepcounter{partCounter}\\}

%
% Various Helper Commands
%

% Useful for algorithms
\newcommand{\alg}[1]{\textsc{\bfseries \footnotesize #1}}

% For derivatives
\newcommand{\deriv}[1]{\frac{\mathrm{d}}{\mathrm{d}x} (#1)}

% For partial derivatives
\newcommand{\pderiv}[2]{\frac{\partial}{\partial #1} (#2)}

% Integral dx
\newcommand{\dx}{\mathrm{d}x}

% Alias for the Solution section header
\newcommand{\solution}{\textbf{\large Solution}}

% Probability commands: Expectation, Variance, Covariance, Bias
\newcommand{\E}{\mathrm{E}}
\newcommand{\Var}{\mathrm{Var}}
\newcommand{\Cov}{\mathrm{Cov}}
\newcommand{\Bias}{\mathrm{Bias}}

\begin{document}

\maketitle

\pagebreak

\begin{homeworkProblem}
    % questions
Make a program for comparing the partitioning of
karate club network. (You can use the following
built‐in functions of networkX.)
\\
\\
a. kernighan\_lin\_bisection\\
b. greedy\_modularity\_communities
% \includegraphics[scale=0.3]{quiz6_1.jpg}

\subsection*{Answer 1}


\begin{python}
import networkx as nx
import matplotlib.pyplot as plt
import numpy as np
import numpy.linalg as LA
from networkx.algorithms.community import greedy_modularity_communities
from networkx.algorithms.community import kernighan_lin_bisection

G = nx.karate_club_graph()
color_map = ['yellow'] * (nx.number_of_nodes(G) - 1)
color_map.append('red')

colors = ['red', 'blue', 'green', 'purple', 'brown', 'yellow']
pos = nx.spring_layout(G)

##################kernighan_lin_bisection
# a tuple with two parts
lst_b = kernighan_lin_bisection(G)
color_map_b = ['black'] * nx.number_of_nodes(G)



# fill in this part
# first_part,secend_part = lst_b
# print(first_part)
# print(secend_part)

color_count = 0
for i in lst_b:
  print("the {} part is ".format(color_count + 1), i )
  for j in i:
    color_map_b[j] = colors[color_count]
  color_count += 1

# draw the first graph
nx.draw_networkx_edges(G, pos)
nx.draw_networkx_nodes(G, pos, node_color=color_map_b)
nx.draw_networkx_labels(G, pos)
plt.axis('off')
plt.show()


######################greedy_modularity_communities
lst_c = list(greedy_modularity_communities(G))
color_map_c = ['black'] * nx.number_of_nodes(G)
# fill in this part

color_count = 0
for i in lst_c:
  print("the {} part is ".format(color_count + 1), i )
  for j in i:
    color_map_c[j] = colors[color_count]
  color_count += 1

nx.draw_networkx_edges(G, pos)
nx.draw_networkx_nodes(G, pos, node_color=color_map_c)
nx.draw_networkx_labels(G, pos)
plt.axis('off')
plt.show()
\end{python}

The result is:
\begin{python}
the 1 part is  {8, 14, 15, 18, 20, 22, 23, 24, 25, 26, 27, 28, 29, 30, 31, 32, 33}
the 2 part is  {0, 1, 2, 3, 4, 5, 6, 7, 9, 10, 11, 12, 13, 16, 17, 19, 21}
\end{python}
\includegraphics[scale=0.3]{quiz10_1.jpg}
\begin{python}
the 1 part is  frozenset({32, 33, 8, 14, 15, 18, 20, 22, 23, 24, 25, 26, 27, 28, 29, 30, 31})
the 2 part is  frozenset({1, 2, 3, 7, 9, 12, 13, 17, 21})
the 3 part is  frozenset({0, 4, 5, 6, 10, 11, 16, 19})
\end{python}
\includegraphics[scale=0.3]{quiz10_2.jpg}






\end{homeworkProblem}
\pagebreak


\end{document}
%
% Non sequential homework problems
%

% Jump to problem 18
