\documentclass{article}

\usepackage{fancyhdr}
\usepackage{extramarks}
\usepackage{amsmath}
\usepackage{amsthm}
\usepackage{amsfonts}
\usepackage{tikz}
\usepackage{graphicx} %插入图片的宏包
\usepackage{float} %设置图片浮动位置的宏包
\usepackage{pythonhighlight}
% \usepackage{subfigure} %插入多图时用子图显示的宏包
% \usepackage[plain]{algorithm}
% \usepackage{algpseudocode}

% \usetikzlibrary{automata,positioning}

%
% Basic Document Settings
%

\topmargin=-0.45in
\evensidemargin=0in
\oddsidemargin=0in
\textwidth=6.5in
\textheight=9.0in
\headsep=0.25in

\linespread{1.1}

\pagestyle{fancy}
\lhead{\hmwkAuthorName}
\chead{\hmwkClass\ : \hmwkTitle}
\rhead{\firstxmark}
\lfoot{\lastxmark}
\cfoot{\thepage}

\renewcommand\headrulewidth{0.4pt}
\renewcommand\footrulewidth{0.4pt}

\setlength\parindent{0pt}


%代码格式设置



%
% Create Problem Sections
%

\newcommand{\enterProblemHeader}[1]{
    \nobreak\extramarks{}{Problem \arabic{#1} continued on next page\ldots}\nobreak{}
    \nobreak\extramarks{Problem \arabic{#1} (continued)}{Problem \arabic{#1} continued on next page\ldots}\nobreak{}
}

\newcommand{\exitProblemHeader}[1]{
    \nobreak\extramarks{Problem \arabic{#1} (continued)}{Problem \arabic{#1} continued on next page\ldots}\nobreak{}
    \stepcounter{#1}
    \nobreak\extramarks{Problem \arabic{#1}}{}\nobreak{}
}

\setcounter{secnumdepth}{0}
\newcounter{partCounter}
\newcounter{homeworkProblemCounter}
\setcounter{homeworkProblemCounter}{1}
\nobreak\extramarks{Problem \arabic{homeworkProblemCounter}}{}\nobreak{}

%
% Homework Problem Environment
%
% This environment takes an optional argument. When given, it will adjust the
% problem counter. This is useful for when the problems given for your
% assignment aren't sequential. See the last 3 problems of this template for an
% example.
%
\newenvironment{homeworkProblem}[1][-1]{
    \ifnum#1>0
        \setcounter{homeworkProblemCounter}{#1}
    \fi
    \section{Problem \arabic{homeworkProblemCounter}}
    \setcounter{partCounter}{1}
    \enterProblemHeader{homeworkProblemCounter}
}{
    \exitProblemHeader{homeworkProblemCounter}
}

%
% Homework Details
%   - Title
%   - Due date
%   - Class
%   - Section/Time
%   - Instructor
%   - Author
%

\newcommand{\hmwkTitle}{Final Report}
\newcommand{\hmwkDueDate}{Feb 6th, 2019}
\newcommand{\hmwkClass}{Essence of Humanities and Social Sciences15:Science and Technology for Society}
\newcommand{\hmwkClassTime}{Section A}
% \newcommand{\hmwkClassInstructor}{Professor Isaac Newton}
\newcommand{\hmwkAuthorName}{\textbf{RUOPENG XU} }
\newcommand{\hmwkAuthorNum}{\textbf{18M38179} }

%
% Title Page
%

\title{
    \vspace{2in}
    \textmd{\textbf{\hmwkClass:\ \hmwkTitle}}\\
    \normalsize\vspace{0.1in}\small{Due\ on\ \hmwkDueDate\ }\\
    % \vspace{0.1in}\large{\textit{\hmwkClassInstructor\ \hmwkClassTime}}
    \vspace{3in}
}

\author{\hmwkAuthorName\\ \hmwkAuthorNum}
\date{}

\renewcommand{\part}[1]{\textbf{\large Part \Alph{partCounter}}\stepcounter{partCounter}\\}

%
% Various Helper Commands
%

% Useful for algorithms
\newcommand{\alg}[1]{\textsc{\bfseries \footnotesize #1}}

% For derivatives
\newcommand{\deriv}[1]{\frac{\mathrm{d}}{\mathrm{d}x} (#1)}

% For partial derivatives
\newcommand{\pderiv}[2]{\frac{\partial}{\partial #1} (#2)}

% Integral dx
\newcommand{\dx}{\mathrm{d}x}

% Alias for the Solution section header
\newcommand{\solution}{\textbf{\large Solution}}

% Probability commands: Expectation, Variance, Covariance, Bias
\newcommand{\E}{\mathrm{E}}
\newcommand{\Var}{\mathrm{Var}}
\newcommand{\Cov}{\mathrm{Cov}}
\newcommand{\Bias}{\mathrm{Bias}}

\begin{document}

\maketitle

\pagebreak

\begin{homeworkProblem}
    % questions
第4回〜第7回(1/23)までの講義で扱った事件や問題のどれか一つを取り上げて,その問題について自分が学者や技術者側の当事者であると想定して(科学者や技術者にも多様な立場がありえるのでそのいずれでも良い),どのように事件や問題の解決に貢献できるかを検討し、2000字程度(英語の場合、約500 words)にまとめよ。

% \includegraphics[scale=0.3]{quiz6_1.jpg}

\subsection*{Answer 1}
% 使用BFS来进行最短路径的问题(从起点到终点),在加权的图上面,找到cost最少的路线
% 初始的点上赋值是0,其他的点都是无穷
% 一个visted set和一个unvisited set. current的点计算他的所有unvisited的(到这个点的距离)+ (从这个点到邻居的距离)。如果比现在的小就更新这个点的距离
% 找到现在身上带着数字的最小的然后把新的点更新到visited


\begin{python}
import networkx as nx
import matplotlib.pyplot as plt
import numpy as np
import sys
import functools
import operator

G = nx.Graph()
G.add_nodes_from(range(0, 5))
G.add_weighted_edges_from([(0, 1, 7), (0, 2, 9), (0, 5, 14), (1, 2, 10), (1, 3, 15), (2, 3, 11), (2, 5, 2), (3, 4, 6), (4, 5, 9)])

plt.figure(figsize=(5, 5))
pos = nx.spring_layout(G)
nx.draw_networkx_edges(G, pos)
nx.draw_networkx_nodes(G, pos)
nx.draw_networkx_edge_labels(G, pos, font_size=16, edge_labels={(u, v): d["weight"] for u, v, d in G.edges(data=True)})
nx.draw_networkx_labels(G, pos)
plt.axis('off')
plt.show()

# find the smallest one
dist_estimate = [sys.maxsize] * nx.number_of_nodes(G) 
# real distance
dist_certainty = [0] * nx.number_of_nodes(G)
dist_estimate[1] = 0


print(dist_estimate)
print(dist_certainty)

#
# please fill in this part (Dijkstra's algorithm)
# calculte the distance from 1 -> 4

n = 0
distance = 0
count = 1
temp_nodes = 0

for visited in dist_certainty:
  if (visited <= 1 ):
# find the smallest estimate in G
    for nodes in G:
      if (dist_certainty[nodes] == 0):
          temp_min = dist_estimate[nodes]
          break
    for node
    s in G:
      if (dist_certainty[nodes] == 0):
        if (dist_estimate[nodes] <= temp_min and dist_estimate[nodes] < 10000):
          temp_min = dist_estimate[nodes]
          temp_node = nodes
    current_distance = temp_min
    current_node = temp_node
      
    print("In step",count,":")
    print("current_distance = ", current_distance)
    print("current_node = ", current_node)

  #     mark this distance as certain
    dist_certainty[current_node] = 1
  #     calculate the distance and compare     
    for neighbor in G.neighbors(current_node):
      if (dist_certainty[neighbor] != 1):
        weight = list(dict.values(G.get_edge_data(current_node,neighbor)))
        current_weight = int(weight[0])
        distance = current_distance + current_weight
        if(distance < dist_estimate[neighbor]):
          dist_estimate[neighbor] = distance

    print("dist_estimate",dist_estimate)
    print("dist_certainty",dist_certainty)
    count = count + 1





#######
# print(dist_estimate)
# print(dist_certainty)
print("From built-in function:")
print(nx.dijkstra_path(G,1,4))
print(nx.dijkstra_path_length(G,1,4))
\end{python}

The result is:
\begin{python}
[9223372036854775807, 0, 9223372036854775807, 9223372036854775807, 9223372036854775807, 9223372036854775807]
[0, 0, 0, 0, 0, 0]
In step 1 :
current_distance =  0
current_node =  1
dist_estimate [7, 0, 10, 15, 9223372036854775807, 9223372036854775807]
dist_certainty [0, 1, 0, 0, 0, 0]
In step 2 :
current_distance =  7
current_node =  0
dist_estimate [7, 0, 10, 15, 9223372036854775807, 21]
dist_certainty [1, 1, 0, 0, 0, 0]
In step 3 :
current_distance =  10
current_node =  2
dist_estimate [7, 0, 10, 15, 9223372036854775807, 12]
dist_certainty [1, 1, 1, 0, 0, 0]
In step 4 :
current_distance =  12
current_node =  5
dist_estimate [7, 0, 10, 15, 21, 12]
dist_certainty [1, 1, 1, 0, 0, 1]
In step 5 :
current_distance =  15
current_node =  3
dist_estimate [7, 0, 10, 15, 21, 12]
dist_certainty [1, 1, 1, 1, 0, 1]
In step 6 :
current_distance =  21
current_node =  4
dist_estimate [7, 0, 10, 15, 21, 12]
dist_certainty [1, 1, 1, 1, 1, 1]
From built-in function:
[1, 2, 5, 4]
21
\end{python}







\end{homeworkProblem}
\pagebreak


\begin{homeworkProblem}
Show all the statuses of current estimates and their certainties while Dijkstra’s algorithm is performed from vertex 0.

\subsection*{Answer 2}
According to the algorithm above, when the Dijkstra is from vertex 0, change dist\_estimate[0] to 0.
\\
The result is:
\begin{python}
In step 1 :
current_distance =  0
current_node =  0
dist_estimate [0, 7, 9, 9223372036854775807, 9223372036854775807, 14]
dist_certainty [1, 0, 0, 0, 0, 0]
In step 2 :
current_distance =  7
current_node =  1
dist_estimate [0, 7, 9, 22, 9223372036854775807, 14]
dist_certainty [1, 1, 0, 0, 0, 0]
In step 3 :
current_distance =  9
current_node =  2
dist_estimate [0, 7, 9, 20, 9223372036854775807, 11]
dist_certainty [1, 1, 1, 0, 0, 0]
In step 4 :
current_distance =  11
current_node =  5
dist_estimate [0, 7, 9, 20, 20, 11]
dist_certainty [1, 1, 1, 0, 0, 1]
In step 5 :
current_distance =  20
current_node =  4
dist_estimate [0, 7, 9, 20, 20, 11]
dist_certainty [1, 1, 1, 0, 1, 1]
In step 6 :
current_distance =  20
current_node =  3
dist_estimate [0, 7, 9, 20, 20, 11]
dist_certainty [1, 1, 1, 1, 1, 1]
From built-in function:
[0, 2, 5, 4]
20
\end{python}
\end{homeworkProblem}
\pagebreak


\begin{homeworkProblem}
Start from vertex 1 and show all the statuses \& their certainties.

\subsection*{Answer 3}
The result is same as the result in Problem 1:
\begin{python}
In step 1 :
current_distance =  0
current_node =  1
dist_estimate [7, 0, 10, 15, 9223372036854775807, 9223372036854775807]
dist_certainty [0, 1, 0, 0, 0, 0]
In step 2 :
current_distance =  7
current_node =  0
dist_estimate [7, 0, 10, 15, 9223372036854775807, 21]
dist_certainty [1, 1, 0, 0, 0, 0]
In step 3 :
current_distance =  10
current_node =  2
dist_estimate [7, 0, 10, 15, 9223372036854775807, 12]
dist_certainty [1, 1, 1, 0, 0, 0]
In step 4 :
current_distance =  12
current_node =  5
dist_estimate [7, 0, 10, 15, 21, 12]
dist_certainty [1, 1, 1, 0, 0, 1]
In step 5 :
current_distance =  15
current_node =  3
dist_estimate [7, 0, 10, 15, 21, 12]
dist_certainty [1, 1, 1, 1, 0, 1]
In step 6 :
current_distance =  21
current_node =  4
dist_estimate [7, 0, 10, 15, 21, 12]
dist_certainty [1, 1, 1, 1, 1, 1]
From built-in function:
[1, 2, 5, 4]
21
\end{python}
\end{homeworkProblem}
\pagebreak


\begin{homeworkProblem}
Explain the reasons why Dijkstra’s algorithm does not work for negative weight edges.

\subsection*{Answer 4}
Because when we think a vertex is visited and change its value in \textit{dist\_certainty} to 1, it will not be visited anymore. If a vertex is in visited, we will think we have already found the shortest path to this vertex. However, this doesn't work in ngetive path. For example, if a graph with negative edge is like this:

\includegraphics[scale=0.3]{quiz9_1.jpg}

When we start from vertex 0, we think the shortest path to 1 is -2, but it is actually -3(from veertex 2 to vertex 1).




\end{homeworkProblem}
\pagebreak


\end{document}
%
% Non sequential homework problems
%

% Jump to problem 18
